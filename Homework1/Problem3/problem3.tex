\chapter{Problem 3}
\prob{
   Let $\mathcal{T}$ be de combinatoric class of the $(n+2)$-polygon triangulation (where $n = 0, 1, 2, \dots$). Find a recursive
    representation of $\mathcal{T}$. Get an functional equation of this representation. Find the counting sequence.
}

\begin{proof}
    Lets make clear one point. These triangulations are ``rooted'', that is, we are going to label one of the polygon edges and
    we are going to label its vertices (let say, as first and second).\pn

    Now lets define a size function to triangulations as the number of triangles that it contains.
    To make it easier lets say that the size of the one edge degenerate triangulation is $0$ (that is, the size of all the triangulations
    of the $2$-polygon). We are going to get rid of this assumption later.\pn
    
    Now, there is an easy way to see these rooted triangulations. Lets represent as $\triangle$ the class containing the unique ``rooted'' 
    triangulation of a triangle. So $\mathcal{T} \times \triangle \times \mathcal{T}$ is a representation of almost all our triangulations.
    The only problem is that this representation always contains at least one triangle. So we can fix this problem by adding the ``no-triangles''
    triangulations class $\mathcal{E}$ that consist of only one element of size $0$. So now we can represent 
    
    \begin{align}
        \mathcal{T} = \mathcal{T} \times \triangle \times \mathcal{T} + \mathcal{E}.
    \end{align}
    
    So, our functional equation should satisfy
    \begin{align}
        T(z) = T(z)^2 z + 1
    \end{align}\pn
    
    Which lead us to
    \begin{align}
        0 = T(z)^2 z - T(z) + 1
    \end{align}\pn
    
    Solving the quadratic for $T(z)$ we get
    \begin{align}
        T(z) = \frac{1 \pm \sqrt{1 - 4z}}{2z}
    \end{align}\pn
    
    From the analysis made in the problem [\ref{problem2}], we already know that $\sqrt{1 - 4z}$ can be written as a polynomial with only
    negative coefficients. But we don't want so, because negative coefficients doesn't mean a thing for us. So from the $\pm$ we are going to
    take the minus symbol and end up with
    
    \begin{align}
        T(z) = \frac{1 - \sqrt{1 - 4z}}{2z}    
    \end{align}\pn
    
    From problem [\ref{problem2}] we know how to write this as a polynomial. And that is
    \begin{align}
            T(z) = \sum_{n=0}^{\infty} \frac{1}{n+1} \binom{2n}{n} z^n
    \end{align}\pn
    
    So, finally, our counting sequence is given by
    \begin{align}
            T_n = \frac{1}{n+1} \binom{2n}{n}
    \end{align}\pn
    
    But we have being counting the 2-polygon triangulation. If we want to get rid of it, our recursive representation will be
    \begin{align}
        \tilde{\mathcal{T}} = \tilde{\mathcal{T}} \times \triangle + \tilde{\mathcal{T}} \times \triangle \times \tilde{\mathcal{T}} + \triangle \times \tilde{\mathcal{T}} + \triangle
    \end{align}\pn
    
    Where $\tilde{\mathcal{T}}$ represents the class of the triangulations with at least one triangle.\pn
    
    Now, the functional equation will be easy to obtain from the previous one. We only want to get rid of the constant term. So
    \begin{align}
        \tilde{T}(z)    &=  T(z)-1  \\
                        &=  \sum_{n=0}^{\infty} \frac{1}{n+1} \binom{2n}{n} z^n - 1 \\
                        &=  \sum_{n=1}^{\infty} \frac{1}{n+1} \binom{2n}{n} z^n. 
    \end{align}\pn
    
    And our counting sequence remains the same than before except that now $T_0 = 0$ and we are done.
\end{proof}